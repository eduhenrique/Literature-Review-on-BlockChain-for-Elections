\documentclass[letterpaper, 12 pt, conference]{ieeeconf} 

\overrideIEEEmargins
% See the \addtolength command later in the file to balance the column lengths
% on the last page of the document

\usepackage{bookmark}
\usepackage[brazil]{babel}
\usepackage[utf8]{inputenc}
\usepackage{hyperref}
\hypersetup{
    colorlinks=true,
    linkcolor=blue,
    filecolor=magenta,      
    urlcolor=cyan,
}

% The following packages can be found on http:\\www.ctan.org
%\usepackage{graphics} % for pdf, bitmapped graphics files
%\usepackage{epsfig} % for postscript graphics files
%\usepackage{mathptmx} % assumes new font selection scheme installed
%\usepackage{times} % assumes new font selection scheme installed
%\usepackage{amsmath} % assumes amsmath package installed
%\usepackage{amssymb}  % assumes amsmath package installed

\title{\LARGE \bf
Levantamento básico sobre Blockchain e segurança para eleições.
}

\author{\\% <-this % stops a space
}


\begin{document}
\maketitle
\thispagestyle{empty}
\pagestyle{empty}

%\begin{abstract}\end{abstract}
\chapter{\textbf{CONCEITOS COMUNS}}
    \subsection{\textit{Bit-commitment}}
        Apresentado por Naor, em \cite{naor1991bit}, o protocolo de \textit{bit-commitment} se tornou o componente básico para diversos outros esquemas. O protocolo é demonstrado com o exemplo de Alice e Bob, onde  Alice escreve a informação (\textit{bit}) e tranca numa caixa, que somente ela tem a chave. A caixa é entregada para Bob (durante o período do \textit{commit}) e quando for necessário ela abre a caixa. Bob sabe que o conteúdo da caixa não foi modificado, visto que estava sobre sua responsabilidade. 
        
        
        
    \subsection{\textit{Blind signature}}
        Em \cite{chaum1984blind} Chaum contextualiza um problema de pagamento aplicando na prática o \textit{Blind signature}. Com a criptografia normal (criptografia assimétrica - com chave pública e privada - esquema de identificação), somente quem paga é coberto pela anonimidade. Com o \textit{Blind signature}, quem recebe o dinheiro também ganha anonimidade. 
        
        Normalmente ocorre a combinação dos dois esquemas, garantindo anonimidade mutua. No geral permite que o usuário obtenha a assinatura de uma mensagem de uma maneira que o assinante não tenha conhecimento da mensagem nem da assinatura resultante. 
        
    \subsection{ Computação \textit{multi-party}}
        Exemplificando com base em \cite{Du:2001:SMC:508171.508174}, o problema de segurança de computação \textit{multi-party} ocorre quando duas ou mais partes desejam estabelecer comunicação que envolva informações privadas e que não há a intenção de torná-las públicas. O problema consiste em como conduzir integração e ao mesmo tempo preservar a privacidade das informações. O \textit{bit-commitment} é utilizado como base.
        
        O problema de segurança de computação \textit{multi-party} trabalha com funções probabilísticas sobre qualquer que seja a informação, em uma rede distribuída, onde cada participante mantém sua entrada de dados, ou parte dela, garantindo independência nessa parte do processo.

    \subsection{Encriptação homomórfica}
        A encriptação homomórfica permite computar e operar dados criptografados sem a necessidade (nem permissão) de decriptá-los. Gentry em \cite{gentry2009fully} discute sobre melhorias nas características da encriptação homomórfica, focando na definição auto-ajustabilidade dos dados, ou \textit{bootstrappable}.
    
    \subsection{Mix-net}
        A técnica foi idealizada por Chaum em \cite{Chaum1981untraceable} e preza pela não necessidade de uma autoridade confiável, onde um participante pode mandar informação para outra parte e se manter anônimo. A ideia é formar uma lista com pseudônimos digitais não rastreáveis. O dono da informação ganha a habilidade de criar assinaturas respectivas ao seu pseudônimo.
        
        Com base no trabalho de Douglas Wikstr̈om \cite{wikstrom2004universally}, o meio seguro é composto pelos mix-servers, trabalhando com uma lista de \textit{cryptotexts} com as informações a serem transmitidas assinadas, exibindo os \textit{cleartexts}, que é a saída da lista de forma randômica.
        Os mix-servers trabalham cooperativamente repetindo o processo de re-encriptar e permutar randomicamente a informação, garantindo a não rastreabilidade.
        
    \subsection{Prova de conhecimento zero}
        Provas de conhecimento zero é uma elegante técnica para limitar a quantidade de informação transferida do dono da informação/prova A para o verificador B. O proponente demonstra a posse da informação sem revelar, computacionalmente, a informação propriamente dita. 
        \cite{goldwasser1989knowledge}. É aplicado para autenticação e para compartilhar e proteger dados.
        
    \subsection{\textit{Diffie-Hellman key exchange}}
        Os protocolos Diffie-Hellman para troca de chaves autenticadas (AKE - \textit{Authenticated Key Exchange}) visa prover uma gama de participante com uma chave secreta compartilhada, a qual deverá ser utilizada para alcançar integridade de mensagens \textit{multicast}, por exemplo. É fundamental para o esquema Diffie-Hellman alcançar o AKE com a autenticação implícita. 
        
        Com o AKE, cada participante tem a garantia de que nenhum outro participante, além do grupo arbitrário de participantes, pode aprender qualquer informação sobre a chave da sessão. Outra característica altamente desejável é a autenticação mútua, onde cada participante tem a certeza de que seus parceiros (ou grupo dos mesmos) realmente possuem a chave da sessão distribuída. \cite{bresson2001provably}
        
    %%bit commitment scheme.
    %% threshold encryption.
    \\
\chapter{\textbf{ARTIGOS REVISADOS}}

\section{A Practical Secret Voting Scheme for Large Scale Elections}\label{I}
    \subsection{Autores}
        Atsushi Fujioka, Tatsuaki Okamoto, Kazuo Ohta.
    \subsection{Destaque}\label{I.B}
        \begin{itemize}
            \item Proposição de um modelo prático de eleição de grande escala, possuindo como participantes, eleitor, administrador/centro de confiança e contador/tabela pública.
            \item Citação de modelos de votação propostos sobre pontos de vista práticos e teóricos.
            \item Separação de dois tipos de abordagens, envio do \textit{ballot} com o voto por um esquema mais criptografia ou por meio de um canal de comunicação anônimo.
        \end{itemize}
    \subsection{Resumo}
    O artigo propõe um esquema de eleição de grande escala. Começa contrapondo dois tipos de abordagens, onde ambas servem para provê segurança no esquema da votação. São elas o envio do voto de forma criptografada e o envio do voto por um canal de comunicação anônimo, logo depois justificando a escolha da segunda abordagem.

    A justificativa para a escolha do canal de comunicação é o fato do voto criptografado trazer sobrecarga para o processamento e para comunicação, visto que é preciso haver uma distribuição da entidade de confiança para prover privacidade ao eleitor. 
    % O problema já é resolvido nos dias de hoje? Há uma mescla entre as duas abordagens?
    
    O autor define etapas para realização do processo. Tais etapas abrangem outras subetapas de segurança além de fluxos principais e fluxos alternativos. As etapas são:
        \begin{itemize}
            \item \textit{Preparation}: Onde a técnica de \textit{blind signature} é utilizada para enviar a mensagem ao administrador. 
            \item \textit{Administration}: O administrador assina a mensagem na qual o \textit{ballot} do eleitor está escondido e então manda de volta.
            \item \textit{Voting}: O eleitor obtém o \textit{ballot} assinado e envia para o contador de forma anônima.
            \item \textit{Collecting}: O contador publica a lista de \textit{ballots} recebidos. 
            \item \textit{Opening}: Opcionalmente, caso ocorra algum problema, o eleitor reivindica seu voto e envia sua chave de encriptação anonimamente por um canal seguro.
            \item \textit{Counting}: Etapa ocorre a contagem dos votos e a publicação do resultado.
        \end{itemize}
        
    \subsection{Observações}
    Sobre a escolha entre as duas abordagens de segurança, o problema não já é resolvido nos dias de hoje? A tecnologia de Blockchain, evidenciada com os Bitcoins, já cuida dessa descentralização da entidade de confiança. Há uma mescla entre as duas abordagens citadas no último item de \ref{I.B}  evidenciada em tecnologias Blockchain como Hyperledger, por exemplo.

\section{An Improvement on a Practical Secret Voting Scheme}
    \subsection{Autores}
        Miyako Ohkubo, Fumiaki Miura, Masayuki Abe, Atsushi Fujioka, and Tatsuaki Okamoto
    \subsection{Destaque}
        O Autor evidencia outras abordagens para aplicação prática de uma eleição eletrônica: computação \textit{multi-party}, encriptação homomórfica, Mix-net e por fim baseado em \textit{blind signature}, com destaque para seção de implementação do canal anônimo com Mix-net. 
    \subsection{Resumo}
        A proposta apresentada pelo autor é baseada no trabalho da seção \ref{I} e sugere uma melhoria para o eleitor por meio de uma nova propriedade que anula a sua necessidade de participar do etapa da contagem além de continuar garantindo as outras propriedades de segurança.
        
        A proposta é manter a etapa da votação segura e composta por um esquema de voto baseado em \textit{blind signature}, além de utilizar a nova propriedade, \textit{walk away}, que possibilita ao eleitor cessar a utilização do sistema após submeter seu voto.
        
        Um \textit{bulletin board} é utilizado para manter as informações, onde qualquer um pode gravar sem identificação explicita e ninguém pode apagar ou sobrescrever informações. Se tratando da criptografia, esquemas já difundidos foram utilizados - \textit{multiparty threshold encryption}, \textit{digital signarute}, \textit{blind signature}.
        
        O esquema proposto nesse trabalho herda a maior parte das propriedades de segurança do trabalho da seção \ref{I}, apresentando como diferença a existência de mais de uma entidade contadora de forma distribuída. Também ocorre o uso da criptografia com \textit{multiparty threshold encryption} ao invés de \textit{bit-commitment} na etapa de obtenção do \textit{ballot}.
        
        A relação entre a identidade do eleitor e o \textit{ballot} é mantida pelo esquema de \textit{blind signature}. O \textit{ballot} é enviado pelo canal seguro de forma irrastreável. Outra etapa diferente da proposta de \ref{I} é a  da contagem, onde os eleitores não precisam mais enviar informação para abrir seu \textit{ballot} por que o contador sabe já sabe decriptar. Essa é a propriedade \textit{walk-away}, definida para os eleitores.
        
\section{An Anonymous Electronic Voting Protocol for Voting Over The Internet}
    \subsection{Autores}
        Indrajit Rayt, Indrakshi Rayt, Natarajan NarasimhamurthiS
    \subsection{Destaque}
        O autor começa destacando as mesmas abordagens citadas em \ref{I.B} e posiciona seu artigo como similar ao da seção \ref{I}, funcionando também para eleições de larga escala sobre a Internet, mas sem a necessidade de um canal anônimo, utilizando um processo similar ao de uma sessão FTP de convidado para a efetivação do voto.
    \subsection{Resumo}
        Os seguintes itens são referenciados como propriedades para a segurança da eleição:
        \begin{itemize}
            \item \textit{Accuracy:}
                Um voto submetido não pode ser alterado, um voto inválido não será contado e cada eleitor tem a garantia de que seu voto será considerado.
            \item \textit{Democracy:}
                Somente um eleitor apto participa do processo e cada eleitor só submete o voto uma vez.
            \item \textit{Privacy:}
                Ao lançar o \textit{ballot} com o voto, não haverá como identificar o eleitor que o submeteu.
            \item \textit{Verifiability:}
                Cada eleitor pode verificar se seu voto.
            \item \textit{No Unauthorized Proxy:}
                Como uma propriedade adicional, se um eleitor apto que recebeu o \textit{ballot} decidir não votar, nenhuma entidade do sistema pode forjar um voto em seu lugar.
        \end{itemize}
        
        Mais características do protocolo aplicado são explanados. Tais características são utilizadas em algum momento durante o processo. A primeira delas é, permutação difícil de inverter. Basicamente é uma randomização de números que para reverter tem um custo de processamento muito alto.
        
        A segunda é o \textit{blind signature} das mensagens. Uma transformação de uma mensagem passível de verificação que só pode ocorrer com a assinatura da entidade responsável, onde qualquer um pode verificar a origem da assinatura. O processo envolve a assinatura da mensagem sem a exposição da informação. Por fim o esquema de encriptação que garante privacidade e imutabilidade da mensagem em transito.
        
        As ações da eleição passam por três diferentes agentes \textit{ballot distributor} (BD), \textit{certifying authority} (CA) e o \textit{vote compiler} (VC). O autor reforça que o processo de não submeter o voto é diferente de se abster, visto que todos eleitores aptos recebem o \textit{ballot} assinado pelo BD. 
        
    \subsection{Observações}
        A proposta serve a uma eleição de grande escala e via internet, entretanto a substituição do canal anônimo com mix-net pelo ftp anônimo pode não ser favorável ao cenário com o voto partindo de dispositivos pessoais. Não é aprofundado as diferenças entre as duas formas pára transporte do voto, não ficando claro se há uma vantagem técnica da abordagem proposta.
        
        Com o meio de transporte por ftp anônimo se torna possível rastrear o endereço IP do dispositivo de onde veio a submissão do voto, se tornando suficientemente seguro se for utilizado centrais ou quiosques para votação, levando a locomoção ao local de voto.
        
\section{Secure E-Voting With Blind Signature}
    \subsection{Autores}
        Subariah Ibrahim, Maznah Kamat, Mazleena Salleh, and Shah Rizan Abdul Aziz
    \subsection{Destaque}
        O trabalho visa utilizar \textit{blind signature} para manter a confidência das informações e a assinatura digital do eleitor para a autenticação. A implementação ocorreu com Java (socket) e para o segurança do canal de comunicação houve a utilização da API de código aberto BouncyCastle \cite{BouncyCastle}.
        
        A chave privada do eleitor, referente a sua assinatura digital, é segura por um esquema de criptografia baseada em senha, com algorítimos SHA (\textit{Secure Hash Algorithms}) e Twofish-CBC, o que resulta no uso exclusivo para eleitores válidos.
    \subsection{Resumo}
        Para evitar os problemas com ações fraudulentas e corruptas, o trabalho apresenta quatro requisitos de segurança:
        \begin{itemize}
            \item Confidencialidade: O \textit{ballot} do eleitor deve ser mantida em segredo, prevenindo uma exposição do voto em relação ao eleitor, o que dificulta também a compra de votos.
            
            \item Integridade: Sem essa propriedade o \textit{ballot} poderia ser adulterado com facilidade, visto que se trata de um ambiente digital online. Para garantir a integridade, o esquema adotado deve garantir que somente votos validos serão considerados, que ninguém posso alterar o voto e se houver alguma tentativa, deve-se expor o responsável.
            
            \item Autenticidade: Deve existir um mecanismo de autenticação que garanta que o eleitor é realmente quem diz ser e que ele esteja apto para participar da eleição. Para que essa propriedade seja satisfeita, durante o período de registo, é preciso atribuir algum tipo de credencial ao eleitor, para que o mesmo possa utilizar durante a etapa da votação. Com essa mesma credencial será possível identificar se o eleitor já submeteu o voto.
            
            \item Verificabilidade: É apresentada sobre duas categorias, individual e universal. A individual é satisfeita quando o eleitor consegue identificar com sucesso que seu voto foi computado e considerado para o resultado final. Já a verificabilidade universal é atribuída à transparência da eleição, quando qualquer um possa auditar as informações publicadas além da possibilidade de verificar se os votos foram contabilizados corretamente.
        \end{itemize}
        
        O trabalho apresenta sete entidades que compõem as etapas do processo: eleitor, administrador, contador, validador, registrador, base do sistema e base nacional com pessoas aptas.
        
    \subsection{Observações}
        A disposição dos participantes da eleição é bem interessante. O fato de existir uma base oficial que representa os participantes aptos para a eleição mostra uma proximidade com o processo real. Esses participantes são eleitores e que podem ser candidatos também, afinal o candidato também vota. Existe também a presença muito importante da entidade da administração que indica o período que a eleição vai ocorrer além de registrar os eleitores para a candidatura, basicamente como pensei para o trabalho do mestrado.
        
\section{Blockchain Voting and its effects on Election Transparency and Voter Confidence}
    \subsection{Autores}
        Teogenes Moura, Alexandre Gomes
    \subsection{Destaque}
        Justificando a falta de confiabilidade nos sistemas convencionais, o autor evidencia as propriedades comuns entre a maioria das blockchais: transparência, imutabilidade, alta disponibilidade, confiabilidade/auditabilidade. É apresentado também o efeito \textit{Winner - Loser}, onde os eleitores que votaram no vencedor tendem a confiar mais no sistema do que os demais. 
        
    \subsection{Resumo}
        Os meios sem blockchain para realização digital de eleições não proveem suficiente nível de transparência para os eleitores, impactando na confiança sobre seu voto permanecer o mesmo até a contagem final dos votos. O artigo discorre sobre blockchain e indica como seu uso pode resolver os problemas de transparência e confiabilidade.
        
    \subsection{Observações}
        O autor posiciona o sistema de eleição somente para ambientes públicos ( consequentemente para amplas votações), lembrando das principais característica da blockchain pública, que são transações baseadas em consenso geral da rede (\textit{trustless}), além de arquitetura distribuída e descentralizada. 
        
        Contudo, talvez o conceito de \textit{trustless} seja modificado ao se tratar de uma blockchain permissionada, visto que normalmente se utilizam outros protocolos para consenso (variantes do BFT), além de ser moldada para o cliente, oferecendo soluções de gerenciamento de identidade que se integram aos sistemas de confiança da empresa. Por ser um ambiente privado não há necessidade de maior rigor em relação aos participantes e o consenso.

\section{The privacy protection mechanism of
Hyperledger Fabric and its application in
supply chain finance}
   \subsection{Autores} Chaoqun Ma, Xiaolin Kong, Qiujun Lan* e Zhongding Zhou
   \subsection{Destaque}
   \subsection{Resumo}
   \subsection{Observações}

\section{Blockchain Technology: Principles and Applications}
   \subsection{Autores} Marc Pilkington
   \subsection{Destaque}
   \subsection{Resumo}
   \subsection{Observações}

%\section{Titulo}
    %\subsection{Autores}
    %\subsection{Destaque}
    %\subsection{Resumo}
    %\subsection{Observações}
    
\chapter{\textbf{CARACTERÍSTICAS COMUMENTE DISCUTIDAS EM SISTEMAS DE ELEIÇÃO}}

\section{Características de implementação}
    \subsection{Abordagens para diminuir a coerção}
        \cite{juels2010coercion} Simplesmente permite a sobreposição do voto, por parte do mesmo eleitor, garantindo que em algum momento o voto ocorrerá com a devida privacidade. No entanto o voto ainda é sujeito ao mesmo tipo de coerção que ocorre nas eleições presenciais, através da compra de votos. 

\section{Características de segurança}
    \subsection{Integridade} 
        Os votos devem ser computados e considerados para o resultado final da mesma forma que foi escolhido pelo eleitor, mantendo a integridade também sobre a quantidade de votos do total de candidatos.
        %counted as cast
        \cite{infoModelWote}
    
    \subsection{Privacidade} 
        A definição 1 de \cite{coney2005PrivacyMeasurement} compõe o conceito empregado de privacidade onde não haveria diferença na informação obtida sobre o voto de algum eleitor caso o mesmo tivesse votado de uma forma diferente. Tal informação é obtida através de saídas disponibilizadas pelo sistema.
        %na liberação de resultado parcial e na quantidade de informação e a forma como ela foi liberada.
        
        Outro ponto para considerar sobre a privacidade é a possibilidade de coerção ou conluio entre participantes da eleição, podendo ou não incluir o próprio eleitor. A privacidade estará garantida caso o sistema impossibilite o próprio eleitor de obter informações que exibam mais do que as saídas regulares do sistema, ou seja, informações que o relacionam com a forma como o voto foi realizado.
        
        \cite{coney2005PrivacyMeasurement}
        %perfect privacy - Lillie Coney, Towards a privacy measurementcriterion for voting systems.
    
    \subsection{Integridade incondicional vs Privacidade incondicional} 
        Segundo \cite{infoModelWote}, existe um \textit{tradeoff} entre os níveis de privacidade e integridade. Por exemplo, um processo eleitoral que declara como vencedor um candidato, independentemente do número de votos, possui zero de integridade e caso não haja saídas do sistema para comprovar o resultado, essa mesma eleição possui o nível máximo de privacidade.
        
        O sistema de eleição consegue estabelecer a relação ideal entre privacidade e integridade quando o voto é computado exatamente da forma que foi processada na etapa de votação e quando o único tipo de saída do sistema é em relação a quantidade dos votos.
        %https://www2.seas.gwu.edu/~poorvi/InfoModelWOTE.pdf  An Information-Theoretic Model of Voting Systems
        
    \subsection{Eleições Verificáveis} 
        O fator de verificabilidade de uma eleição pode ser resumida pela capacidade de validação sobre o resultado do algoritmo da fase de contagem dos votos. Mesmo que a integridade do sistema não seja perfeita, se existir a capacidade de identificar que houve uma irregularidade na contagem dos votos então a característica de verificabilidade está garantida.
        \cite{infoModelWote}
    %% Cast As Intended, Recorded asCast, and Counted as Recorded
    
    \subsection{cryp-tographic voting protocols}
    %Fazer depois do projeto informacional
        \cite{adida2008helios}
        %heliosvoting paper

\chapter{\textbf{HYPERLEDGER - DEFINIÇÕES}}
%% compilação com informações das diferentes Ledgers

\section{\textit{Distributed Ledgers}} 
    \subsection{Hyperledger Besu} %https://www.hyperledger.org/blog/2019/08/29/announcing-hyperledger-besu
    
        É um client Ethereum, feito para ser \textit{enterprise-friendly} tanto para redes públicas quanto para permissionadas. Hyperledger Besu conta com diversos algoritmos de consenso, incluindo \textit{Proof of Work} and \textit{Proof of Authority}.
    
        Inclui também um CLI assim como API para operações nos nós e na rede Ethereum. O cliente API do Besu suporta normalmente principais funcionalidades do Ethereum, como contratos inteligentes e desenvolvimento de Apps descentralizados. Não há suporte para o gerenciamento de chaves no cliente, devido a questões de segurança. Para suprir essa necessidade o EthSigner ou outras tecnologias do Ethereum podem ser utilizadas.
    
\section{Funções de segurança}
    \subsection{Privacidade dos dados}%The privacy protection mechanism of Hyperledger Fabric and its application in supply chain finance
        A característica de destaque em relação a privacidade das informações é a permissão de acesso aos dados divididos por canais. Os \textit{peers} de um mesmo canal compartilham a \textit{ledger}, um novo \textit{peers} precisa do reconhecimento do canal para poder ingressar no canal e então transacionar com os outros. 
    
        O segundo nível de privacidade se encontra com o \textit{private data collection}, que serve para controlar quais organizações inseridas na blockchain tem acesso às informações daquele canal.
  
    \subsection{Integridade dos dados} %https://hyperledger-fabric.readthedocs.io/en/release-1.4/arch-deep-dive.html
        O ponto forte da blockchain em relação a integridade dos dados é encontrado em sua essência. O bloco com as transações possuem uma referencia para o bloco anterior. Caso alguma informação já difundida seja modificada, toda a blockchain a partir daquele ponto será comprometida. 
    
        Para os dados recém requisitados, o cliente de posse da transação e da aprovação do \textit{endorsing peer} faz uma requisição \textit{broadcast}, que tem como destino os \textit{peers} de um determinado canal de comunicação. O \textit{orderer} se encarrega de propagar a transação aos \textit{peers}. 
    
        Cada \textit{peer} que recebeu a requisição e que estiver com o ultimo estado da \textit{ledger} irá verificar se o endorso está de acordo com a politica do \textit{chaincode} e se a transação possui uma chave de acordo com o estado atual da \textit{ledger}. Caso passe nessa validação, a transação é marcada como válida, caso contrário, inválida. Então cada \textit{peer} inclui o novo registro em sua cópia da blockchain e da \textit{ledger} e \textbf{(caso a transação foi válida?)} o estado atual é atualizado.
    
    \subsection{Não rastreabilidade do eleitor (anonimidade vs verificabilidade)} %https://hyperledger-fabric.readthedocs.io/en/release-1.4/identity/identity.html
        Para acessar a blockchain, um usuário necessitará possuir uma identidade emitida por uma entidade certificadora da rede. Para permitir que uma identidade registre informações na rede, o \textit{Membership service provider (MSP)} utiliza o componente \textit{Public key infrastructure}(PKI), empregando conceitos de criptografia assimétrica. O registro do voto na rede blockchain garantirá a anonimidade do eleitor mas garantindo a integridade do voto, permitindo ao mesmo ser considerado para o resultado total da eleição. 
    
    \subsection{Verificabilidade dos votos}
        Uma vez que toda alteração na \textit{ledger} é acessível para quem tem acesso ao canal onde as transações ocorrem, a única forma de não poder auditar os votos é se o usuário não tiver permissão para ver os registros daquela organização ou daquele canal como um todo. Esse acesso respeita os diversos níveis de configuração e de segurança para os dados.
    
    \subsection{Verificabilidade/Confiabilidade no resultado de eleição}
        O resultado da eleição está de acordo com a verificabilidade dos votos. O resultado tende a ter o mesmo nível de confiança para os eleitores, independente se seu candidato obteve o resultado esperado na eleição ou não. Como os dados possuem um nível transparência enorme, questionamentos duvidosos ou de má índole perdem embasamento, resultando num único nível de confiança entre os usuários.      


    
\bibliographystyle{unsrt}%abbrv
\bibliography{references}

\end{document}